\documentclass{article}
\usepackage[utf8]{inputenc}
\usepackage{graphicx}
\usepackage{tabularx}
\usepackage{booktabs}
\usepackage{hyperref}
\usepackage{listings}
\usepackage{amsmath}
\usepackage{mathtools}
\usepackage[margin=0.85in]{geometry}
\usepackage[backend=bibtex,style=ieee]{biblatex}
\usepackage{newunicodechar}
\newunicodechar{fi}{fi}
\addbibresource{textmining_julki092.bib}
\newcommand{\HRule}[1]{\rule{\linewidth}{#1}}

\begin{document}
	
	\title{\textsc{732A92 Text Mining} \\ [2.0cm]
		\HRule{0.5pt} \\
		\LARGE \textbf{\uppercase{Classifying Stock Price Movements based on 8-K SEC filings: A comparison of Recurrent Neural Networks and Gradient Boosting Decision Trees}}
		\HRule{2pt} \\ [0.5cm]
		\normalsize \today \vspace*{5\baselineskip}}
	
	\date{}
	
	\author{
		Name: Julius Kittler \\ 
		Student ID: julki092 \\ 
		Link\"{o}ping University}
	
	\maketitle
	\newpage
	
	\begin{abstract}
		
	This text mining project makes use of filings from the U.S. Securities and Exchange Commission (SEC), in particular 8-K filings, to forecast short-term percentage changes in stock prices. 8-K filings have to be published by public companies in the U.S. for certain defined, business-relevant events. The forecasting problem is approached as a classification problem with five classes: large decrease, small decrease, no change, small increase and large increase (of the stock price before and after filing date of the 8-K filing). Instead of using existing data sets, a new data set for the first quarter of 2019 is created. Furthermore, this project makes a contribution by comparing the previously successful methods for this problem: recurrent neural networks (RNNs) with word embeddings (GloVe) and gradient boosting decision trees (GBDT) with bag-of-words. 
	
	\end{abstract}

	\newpage
	\tableofcontents
	\newpage
	\listoffigures
	\listoftables
	\newpage

	\section{Introduction}
	
	Forecasting stock prices has been a relevant problem since the existence of publicly traded companies. Today, it is an even more relevant problem than ever before because we have the technological infrastructure to put our forecasts to practice in form of automatic trading systems. Not only can we obtain massive amounts of financial data via APIs, but we can also execute trades via APIs. With commission free trading becoming the industry standard in the U.S., executing trades via APIs is even offered for free by some companies nowadays \cite{noauthor_alpaca_nodate}. 

	\subsection{SEC Filings}
	
	The SEC is a government agency in the United States with the mission to "protect investors, maintain fair, orderly, and efficient markets, and facilitate capital formation" \cite{noauthor_sec.gov_nodate}. An important task of the SEC is to ensure that publicly traded companies inform their shareholders and the public about their business.
	
	For instance, the SEC requires companies to publish their quarterly and annual results, and inform shareholders about certain relevant events. For each of these purposes, companies have to file specific documents. For instance, the annual report corresponds to the 10-K, the quarterly report corresponds to the 10-Q and another report for specific relevant events corresponds to the 8-K filing.
	
	Importantly, SEC filings are actively used by traders when making investment decisions. Many trading platforms such as Webull and thinkorswim also provide traders with the recent SEC filings of any tradable company (along with other information such as fundamental data, news data and historical prices). SEC filings are interesting for stock price forecasting because they are standardized, publicly accessible for free and because they contain relevant, objective and generally accurate information.
	
	
	\subsection{8-K Filings}
	
	Companies need to publish an 8-K filing for major events relevant for their business. For instance, such events might be a change in the board of directors, a potential delisting from a stock exchange or a merger. To be precise, there are 31 different 8-K filing events from 9 different sections. One 8-K filing may contain information for several such events. Every 8-K filings clearly states for which events it contains information. A complete list of all events and sections, taken from the official SEC website \cite{noauthor_sec.gov_nodate-1}, is shown in the table ~\ref{table:8kevents}.
	
	In general, 8-K filings are  due within four business days after the event \cite{kenton_8-k_nodate}, a relatively short time period. Because 8-K filing correspond to major events for the company and because they need to be published shortly after an event occurred, 8-K filings seem interesting for predicting short-term volatility in the stock market. Moreover, the important information in 8-K filings is generally represented in form of text data, whereas other filings such as the annual report often focus on numerical data represented in tabular form. Text data is relatively simple to extract from HTML documents in order to generate features for training machine learning models (compared to tabular and graphical data).
	
	To see some examples of 8-K filings, one may go to the official  \href{https://www.sec.gov/cgi-bin/browse-edgar?company=&CIK=&type=8-K&owner=include&count=40&action=getcurrent}{SEC website}. In particular, three examples can be found here: \href{https://www.sec.gov/ix?doc=/Archives/edgar/data/1339947/000119312519299728/d840037d8k.htm}{example 1}, \href{https://www.sec.gov/Archives/edgar/data/883975/000149315219018330/form8-k.htm}{example 2}, \href{https://www.sec.gov/Archives/edgar/data/1419275/000118518519001650/greenbox20191125_8k.htm}{example 3}.
	
	\begin{table}[h!]
	\centering
	\caption{Overview of all 8-K sections and events}
	\label{table:8kevents}
	
	\begin{tabularx}{\textwidth}{|X|l|X|}
		\toprule
		&      &                                   \\
		Section & Item & Event \\
		\midrule
		Registrant's Business and Operations & 1.01 & Entry into a Material Definitive Agreement \\
		& 1.02 & Termination of a Material Definitive Agreement \\
		& 1.03 & Bankruptcy or Receivership \\
		& 1.04 & Mine Safety - Reporting of Shutdowns and Patterns of Violations \\
		Financial Information & 2.01 & Completion of Acquisition or Disposition of Assets \\
		& 2.02 & Results of Operations and Financial Condition \\
		& 2.03 & Creation of a Direct Financial Obligation or an Obligation under an Off-Balance Sheet Arrangement of a Registrant \\
		& 2.04 & Triggering Events That Accelerate or Increase a Direct Financial Obligation or an Obligation under an Off-Balance Sheet Arrangement \\
		& 2.05 & Costs Associated with Exit or Disposal Activities \\
		& 2.06 & Material Impairments \\
		Securities and Trading Markets & 3.01 & Notice of Delisting or Failure to Satisfy a Continued Listing Rule or Standard; Transfer of Listing \\
		& 3.02 & Unregistered Sales of Equity Securities \\
		& 3.03 & Material Modification to Rights of Security Holders \\
		Matters Related to Accountants and Financial Statements & 4.01 & Changes in Registrant's Certifying Accountant \\
		& 4.02 & Non-Reliance on Previously Issued Financial Statements or a Related Audit Report or Completed Interim Review \\
		Corporate Governance and Management & 5.01 & Changes in Control of Registrant \\
		& 5.02 & Departure of Directors or Certain Officers; Election of Directors; Appointment of Certain Officers; Compensatory Arrangements of Certain Officers \\
		& 5.03 & Amendments to Articles of Incorporation or Bylaws; Change in Fiscal Year \\
		& 5.04 & Temporary Suspension of Trading Under Registrant's Employee Benefit Plans \\
		& 5.05 & Amendment to Registrant's Code of Ethics, or Waiver of a Provision of the Code of Ethics \\
		& 5.06 & Change in Shell Company Status \\
		& 5.07 & Submission of Matters to a Vote of Security Holders \\
		& 5.08 & Shareholder Director Nominations \\
		Asset-Backed Securities & 6.01 & ABS Informational and Computational Material \\
		& 6.02 & Change of Servicer or Trustee \\
		& 6.03 & Change in Credit Enhancement or Other External Support \\
		& 6.04 & Failure to Make a Required Distribution \\
		& 6.05 & Securities Act Updating Disclosure \\
		Regulation FD & 7.01 & Regulation FD Disclosure \\
		Other Events & 8.01 & Other Events \\
		Financial Statements and Exhibits & 9.01 & Financial Statements and Exhibits \\
		\bottomrule
	\end{tabularx}

	\end{table}%	

	\subsection{Stock Price Forecasting with 8-K filings}
	
	This section gives a short overview of previous work about forecasting stock prices with 8K filings. Since the focus of this project is classification, this literature review also focuses on previous work with classification. There are few public research papers that made use of 8-K filings for stock price forecasts. However, the papers that are publicly available show promising results. 
	
	For instance, Lee et al. could achieve an increase in accuracy by 10 percent when including text data from 8-K filings into a baseline model that only used financial metrics \cite{lee_importance_2014}. This study was solving a classification task with three classes: price increase ($> 1\%$), price decrease ($< -1\%$) or no relevant change ($< |1 \%|$). Compared to a random classification with 33 percent accuracy, the best model of the study could achieve a 55 percent accuracy on the test data. The main model used for this study was a random forest classifier because it outperformed other models such as multi layer perceptron and logistic regression. Unigram features of the text data were used along with non-negative matrix factorization for dimensionality reduction.
	
	Another study by Saleh et al. extended the research by Lee et al. \cite{saleh_neural_nodate}. In addition to the data from the 8-K filings, the researchers used text data from Twitter. Furthermore, they used convolutional neural networks (CNNs) and recurrent neural networks (RNNs) instead of random forests. Again, with three possible classes, a random classification would have given a 33 percent accuracy. The best test accuracies were 51 percent without and 52.6 percent with the Twitter data. Instead of unigram features, word embeddings from GloVe were used.
	
	Holowczak et al. compared the performance of various common algorithms (random forest, naive bayes, support vector machine, k-nearest neighbor and ridge classifier) on a binay classification task (price increase vs. decrease) \cite{holowczak_testing_2019}. However, only 8-K filings for a particular event, item 4.01, were used. The best results were achieved with a linear support vector classifier with a classification accuracy of 54.4 percent on the test data. The text data had been transformed with frequency-inverse document frequency (tf-id) vectorization.
	
	Sakarwala et al. also conducted a binary classification  (price increase vs. decrease) \cite{sakarwala_use_2019}. The best result with an accuracy of 68 percent was obtained with CNNs and RNNs. Again, word embeddings from GloVe were used.


	%Other research papers investigate different aspects of 8-K filings instead of using them for stock price forecasts. For instance, there is research for predicting content in 8-K filings based on their text \cite{lee_predicting_2008} and for understanding the relation of 8-K filings and analysts' information \cite{gleason_selective_nodate}.%
	
	%Decision support from financial disclosures with deep neural networks and transfer learning --> exclude because they did not do classification.%
	
	%Predicting Stock Returns by Automatically Analyzing Company News Announcements --> exclude because of only few items and seemingly focus on topic modeling rather than classification%
	
	%https://medium.com/@babbemark/bert-is-the-word-predicting-stock-prices-with-language-models-8d5205b8537c --> exclude because unprofessional%
	
	
	%subsection{Stock Price Forecasting with Text Data}%
	%Content of Annual Reports as a Predictor for Long Term Stock Price Movements%
	
	%\subsection{Stock Price Forecasting in General}%
	%Stock Market Analysis: A Review and Taxonomy of Prediction Techniques%


	\subsection{Research Questions}
	
	There are mainly three research questions that this report aims to address. The focus is not only on the stock price classification itself but importantly also on understanding how the model works and on evaluating whether the best model could be used in practice.
	
	\begin{enumerate}
		\item Research question: Does a RNN with LSTM or the GBDT Catboost perform better at classifying stock price movements based on text data from 8-K filings?
		\item Research question: How can the best model be interpreted: In which scenario does it perform better or worse (event of the 8-K filing, industry and location the company)? What are the most important features? 
		\item Research question: Could the best model be profitable when used in an automatic trading strategy?
	
	\end{enumerate}

	\section{Theory}
	
	\subsection{Na\"ive Bayes}
	
	tbd
	
	\subsection{Gradient Boosting Decision Tree (GBDT)}
	
	Gradient boosting with decision trees became popular with the development of XGBoost in 2016 \cite{chen_xgboost:_2016} and lightGBmM in 2017 \cite{ke_lightgbm:_2017}. In 2018, another GBDT library called Catboost was published, which claims to handle categorical features particularly well. Furthermore, Catboost was shown to outperform XGBoost and lightGBM on various datasets with default parameters. For these reasons, Catboost is used in this project. The implementational details of Catboost are rather involved and are therefore not fully covered in this report. Instead, what follows is a short description of how GBDT work in general and how Catboost is different from other GBDT implementations.
	
	
	\subsection{Long Short-Term Memory (LSTM)}
	
	tbs
	
	\section{Data}
	
	\subsection{Sources}
	
	The data used for this text mining project comes from a variety of sources. The stock price data (with daily resolution) was retrieved with the financial API \href{https://www.tiingo.com}{Tiingo}. The SEC filings were downloaded from \href{https://www.sec.gov/Archives/edgar/full-index/}{EDGAR}, the official archive for SEC filings. The overview of companies by CIK, necessary to merge the SEC filings with the stock prices, was taken from the service \href{http://rankandfiled.com/#/data/tickers}{Ranked and Filed}. The industry categorization (SIC domain) for the companies was taken from \href{https://siccode.com}{SICCODE}. Lastly, the overview of the 8-K events, used to extract the 8-K events from each filing, was taken from the  \href{https://www.sec.gov/fast-answers/answersform8khtm.html}{SEC documentation}. 
	
	\subsection{Retrieval}
	
	The dataset used for training the models was created with the following steps: 
	
	\begin{enumerate}
		\item A list of all 8-K filings for the first quarter of 2019 was retrieved from EDGAR. This list contained a total of 16449 8-K filings, including the CIK number (to identify the company) and the filing date.
		\item The list was merged with the data from Ranked and Filed to get the ticker symbol, exchange market and SIC number (representing the industry) of the companies corresponding to each 8-K filing.
		\item All  8-K filings from companies that were not listed on the NASDAQ, the NYSE or the AMEX (according to Ranked and Filed) were removed. In particular, the removed 8-K filings corresponded to companies listed on NYSE ARCA, OTC or OTCBB. The resulting list contained a total of 9318 8-K filings.
		\item All 8-K filings, for which no stock price data was available from Tiingo were removed from the list. The resulting shortlist contained a total of 8030 8-K filings. 
		\item For the 8030 8-K filings, the stock price data was extracted from the Tiingo data, making use of the filing date from EDGAR. For each filing, the percentage change from the closing price of the day before the filing date and the open price of the day after the filing date was computed as target variable.
		\item The 8030 8-K filings from the shortlist were downloaded from EDGAR. When processing the data (see below) 8-K filings with more than 1 Mio. characters were removed due to file size limitations of the libraries that were used for processing. This left a total of 7975 8-K filings for training the model.
	\end{enumerate}
	
	\subsection{Processing}
	
	Each raw 8-K filing, a text file containing HTML code, was processed as follows. First, graphics and embedded PDFs were removed and HTML tags were removed as well. Second, the resulting text data was tokenized with the natural language processing library spaCy, using the English language model \lstinline{en_core_web_sm}. Third, stop words, non-alphabetical tokens and tokens with only one character were removed. Fourth, the remaining tokens were lemmatized with spaCy. Later on, very rare and frequent tokens were removed as well. This will be covered in the section about hyperparameter tuning. 
	
	
	\subsection{Descriptive Statistics}
	
	
	\begin{figure}[h]
		\includegraphics[width=\linewidth]{img/test.png}
		\caption{A boat.}
		\label{fig:boat1}
	\end{figure}
	
	Figure \ref{fig:boat1} shows a boat.
	
	
	\subsubsection{General}
	
	\subsubsection{Target Variable}
	
	The target variable is computed for each 8-K filing as follows. Let the company corresponding to the 8-K filing be company C.
	
	The percentage change from the closing stock price of the day before the filing date and the open price of the day after the filing date is computed.
	
	\subsubsection{Feature Variables}
	

	\section{Method}
	
	\subsection{Loss Function}
	
	For multi-class classification problems, the softmax function is generally used as loss function (in neural networks and in gradient boosting trees). Hence, it is also used for this project. However, the loss function differs slightly between the two models.
	
	\subsubsection{GBDT}
	
	The loss function used for Catboost is called \lstinline{MultiClass} loss function and is essentially a weighted softmax function \cite{noauthor_multiclassification:_nodate}. It is computed as a scalar value across all observations and classes. The definition is as follows:
	
	\begin{align}
	L = \frac{\sum_{i = 1}^N w_i \log \left(\frac{\exp a_{i, t_i}}{\sum_{j = 0}^{M-1} \exp a_{i, j}}\right)}{\sum_{i = 1}^N w_i}, 
	\end{align}
	
	where $i = 1, ..., N$ is the index for the observation, $j = 0, ..., M-1$ is the index of the class, $a_{i, j}$ is the model output for observation $i$ and class $j$, $w_i$ is the weight assigned for observation $i$ during training, and $t_i$ is the index of the true class of observation $i$ so that $a_{i, t_i}$ is the model output for the true class of observation $i$.
	
	Essentially, the term in the parentheses is the softmax function, which returns a value between 0 and 1 for each observation $i$ to belong to its true class $t_i$. If the model performs well for observation $i$, then the softmax value will be large. For instance, if the value is 1, this means that according to the model, we have a probability of 1 that observation $i$ is from class $t_i$, i.e. from the true class of the observation. In other words, the model would perfectly identify the true class of observation $i$. Since the log is taken, and since $log(1)=0$, the loss incurred for this perfectly classified observation will be 0. If the softmax value is smaller than 1, the loss for an observation will be negative and will hence affect the training process.
		
	\subsubsection{LSTM}
	
	The loss function used for the LSTM is similar to the loss function from Catboost. However, the observations $i$ are not weighted by any particular weights $w_i$. Essentially, a softmax function is used in the output layer so that the model directly outputs probabilities for each observation $i$ to belong to class $j$. Then, the loss function \lstinline{CategoricalCrossentropy} is used in \lstinline{TensorFlow} \cite{noauthor_module:_nodate}. For a multi-class problem, the categorical cross-entropy is defined as follows:
	
	\begin{align}
	L = \frac{\sum_{i = 1}^N \log \left(\frac{\exp a_{i, t_i}}{\sum_{j = 0}^{M-1} \exp a_{i, j}}\right)}{N}, 
	\end{align}
	
	where the notation is the same as in the loss function for GBDT. Note that in other notations this loss function usually has another sum over the classes along with an indicator function that is multiplied with the log (implying whether observation $i$ is from class $j$). However, here $a_{i, t_i}$ is defined as the model output for the true class of observation $i$ and hence the indicator function is not needed.
	

	% https://deepnotes.io/softmax-crossentropy
	

%	$$P(y = c | \textbf{x}) = \frac{\exp{z_c}}{\sum_{k = 1}^K \exp{z_k}}$$
	
	 %$\textbf{x}$ is a feature vector for one observation, $c$ is its corresponding class index, $K$ is the total number of classes, and $z_c$ is a real number given as output by the model. For instance, with a linear model, we might have $z_c = \textbf{x}^T\textbf{w}_c + b_c$. Basically, the model output $z_c$ is normalized by the softmax function to get a value in the interval from 0 to 1. This value is then interpreted as probability that observation $\textbf{x}$ corresponds to class $c$. A classification for observation $\textbf{x}$ can be obtained by taking $\underset{c}{\mathrm{argmax}} \ P(y = c | \textbf{x})$.
	
	\subsection{Training, Testing, Validation}
	
	\subsection{Hyperparameter Tuning}
	
	\subsection{Evaluation}
	
	\subsubsection{Research Question 1: Metrics}
	
	For evaluating the results, confusion matrix, accuracy and class-wise precision are reported. The precision is particularly important for considering potential potential trading strategies based on the model.
	
	Class-wise precision is computed for each class individually. For a given class c, the class-wise precision is defined as the number of observations of class c that were classified as class c (TP for true positive), divided by the number of any observations that were classified as class c, regardless of whether they are of class c or not (TP + FP for the sum of all true positives and false positives).
	
	\begin{align}
	\mathrm{precision}_c = \frac{TP_c}{FP_c + TP_c}
	\end{align}
	
	Precision is chosen as prioritized evaluation metric in order to reflect the assumptions that we want to avoid loosing money in a trade and that missing out on a good trade is acceptable. To explain this, it is best to consider an example. Assume that we take a long position whenever the model classifies a large stock price increase (\verb|lg_inc|), i.e. we buy stocks expecting that their stock price will increase. If the stock price decreases instead, we would loose money. To avoid this loss, we ideally want that among all cases where the model classifies a large increase (FP + TP), the stock price actually increases (TP). This means that we care about the precision. 
	
	In addition to reporting the evaluation metrics numerically, they are also visualized by certain categorical variables (e.g. industry of the company, event of the 8-K filings, location of the company) in order to identify issues as well as future potentials of the models. 
	
	\subsubsection{Research Question 2: Model Interpretation}
	
	tbd
	
	\subsubsection{Research Question 3: Backtesting}
	
	tbd
	
	\section{Results}
	
	Table ~\ref{table:prosConsOptionalApproaches} summarizes
	the benefits and drawbacks (``Pros and Cons'') of each 
	approach.
	
	\begin{table}[h]
	\centering
	\caption{The pros and cons of Scala's optional classes}
	\label{table:prosConsOptionalApproaches}
	
	\begin{tabular}{llrrrrr}
		\toprule
		&        & \multicolumn{5}{l}{Predicted} \\
		&        &    lg\_dec & sm\_dec & zero & sm\_inc & lg\_inc \\
		\midrule
		True & lg\_dec &       107 &     46 &   53 &     32 &     78 \\
		& sm\_dec &        39 &     56 &  117 &     49 &     48 \\
		& zero &        34 &     54 &  146 &     46 &     33 \\
		& sm\_inc &        55 &     59 &  102 &     65 &     43 \\
		& lg\_inc &        82 &     42 &   41 &     40 &    128 \\
		\bottomrule
	\end{tabular}
	\end{table}%		

	\section{Discussion}
	
	\section{Conclusion}
	
	\section{References}
	
\printbibliography
\end{document}